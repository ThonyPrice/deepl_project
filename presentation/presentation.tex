\documentclass{beamer}
\usepackage[utf8]{inputenc}

\usetheme{Copenhagen}

\title{
  Classification of ImageNet \\
  with Convolutional Neural Networks
}
\author{
  Niklas Lindqvist,\\
  Thony Price,\\
  William Skagerström\\
}

\begin{document}
\maketitle

\begin{frame}
  \frametitle{Agenda}

  \begin{itemize}
    \item Proposal and intention
    \item Implementation
    \item Results
  \end{itemize}
\end{frame}

\section{Proposal and intention}
\begin{frame}
  \frametitle{Project proposal}
  \begin{itemize}
	  \item Implement a CNN
    \item Achieve acceptable results in terms of accuracy (on ImageNet)
    \item Apply techniques from course to optimize performance of network
  \end{itemize}
\end{frame}

\section{Implementation}
\begin{frame}
  \frametitle{Research of the field}
  We began looking at state of the art
  \begin{itemize}
    \item GoogLeNet
    \item RezNet
    \item VGGNet
  \end{itemize}

  Then to more achievable en par with our scope. Here Stanford's course \textbf{CS231} proved a valuable source. There was many reports on classification of ImageNet by CNNs.
\end{frame}

\begin{frame}
  Considering this we wen't for an VGGNet inspired architecture of the network. The leftmost represents the most vanilla. Each step to the right a optimization technique is added.
  \textbf{Insert table of architecture here}
\end{frame}

\begin{frame}
  We initialized a run on Google Cloud and the result was... OVERFITTING
  \textbf{Insert plots here}
\end{frame}

\begin{frame}
  To battle this we applied
  \begin{itemize}
    \item Batch normalization (All layers)
    \item Dropout (Dense layers)
    \item L2 regularization (X layers)
  \end{itemize}
  Tinkering with these variables resulted in improved performance \textbf{insert plot here/next slide}
\end{frame}



\end{document}
